\begin{flushleft}
    % Ubah kalimat berikut sesuai dengan nama departemen dan fakultas
    \textbf{Departemen Teknik Komputer - FTEIC}\\
    \textbf{Institut Teknologi Sepuluh Nopember}\\
  \end{flushleft}
  
  \begin{center}
    % Ubah detail mata kuliah berikut sesuai dengan yang ditentukan oleh departemen
    \underline{\textbf{EC184701 - PRA TUGAS AKHIR (2 SKS)}}
  \end{center}
  
  \begin{adjustwidth}{-0.2cm}{}
    \begin{tabular}{lcp{0.7\linewidth}}
  
      % Ubah kalimat-kalimat berikut sesuai dengan nama dan NRP mahasiswa
      Nama Mahasiswa &:& Rifqi Alhakim Hariyantoputera \\
      NRP &:& 07211840000055 \\
  
      % Ubah kalimat berikut sesuai dengan semester pengajuan proposal
      Semester &:& Ganjil 2021/2022 \\
  
      % Ubah kalimat-kalimat berikut sesuai dengan nama-nama dosen pembimbing
      Dosen Pembimbing &:& 1. Dr. Eko Mulyanto Yuniarno, S.T., M.T. \\
      & & 2. Reza Fuad Rachmadi, S.T., M.T., Ph.D. \\
  
      % Ubah kalimat berikut sesuai dengan judul tugas akhir
      Judul Tugas Akhir &:& \textbf{Smart Whiteboard: Pengenalan Teks Huruf Balok } \\
      & & \textbf{Menggunakan You Only Look Once (YOLO)} \\
  
      Uraian Tugas Akhir &:& \\
    \end{tabular}
  \end{adjustwidth}
  
  % Ubah paragraf berikut sesuai dengan uraian dari tugas akhir
  Tulisan merupakan salah satu bentuk ragam komunikasi. Dengan adanya tulisan, suatu ide atau gagasan dapat dituangkan dan dan diabadikan. Segmentasi dokumen gambar menjadi suatu bentuk kalimat merupakan Langkah penting untuk memahami suatu dokumen secara utuh. Walaupun diklaim memiliki akurasi hingga 99\%, Optical Character Recognition (OCR) yang ada saat ini memiliki penurunan akurasi ketika dihadapkan kepada gambar dengan kualitas rendah. Berbagai teknologi Internet of Things diterapkan pada papan tulis sehingga memiliki fitur tambahan yang dapat mempermudah proses belajar dan mengajar. Telah banyak teknologi yang dapat memproyeksikan gambar atau tulisan pada computer ke papan tulis pintar. Namun, belum ada teknologi yang mampu mengenali tulisan tangan pada papan tulis pintar. Maka dari itu, diperlukannya suatu algoritma atau program untuk mendeteksi dan klasifikasi teks huruf balok untuk diterapkan pada alat Smart Whiteboard. Sehingga didapatkan tujuan akhir yaitu untuk membuat program komputer yang dapat melakukan pengenalan teks huruf balok menggunakan YOLO sehingga dapat diimplementasikan pada alat Smart Whiteboard.
    
  \vspace{1ex}
  
  \begin{flushright}
    % Ubah kalimat berikut sesuai dengan tempat, bulan, dan tahun penulisan
    Surabaya, Desember 2021
  \end{flushright}
  \vspace{1ex}
  
  \begin{center}
  
    \begin{multicols}{2}
  
      Dosen Pembimbing 1
      \vspace{12ex}
  
      % Ubah kalimat-kalimat berikut sesuai dengan nama dan NIP dosen pembimbing pertama
      \underline{[Dr. Eko Mulyanto Yuniarno, S.T., M.T.]} \\
      NIP. 196806011995121000
  
      \columnbreak
  
      Dosen Pembimbing 2
      \vspace{12ex}
  
      % Ubah kalimat-kalimat berikut sesuai dengan nama dan NIP dosen pembimbing kedua
      \underline{[Reza Fuad Rachmadi, S.T., M.T., Ph.D.]} \\
      NIP. 198504032012121000
  
    \end{multicols}
    \vspace{6ex}
  
    Mengetahui, \\
    % Ubah kalimat berikut sesuai dengan jabatan kepala departemen
    Kepala Departemen Teknik Komputer FTEIC - ITS
    \vspace{12ex}
  
    % Ubah kalimat-kalimat berikut sesuai dengan nama dan NIP kepala departemen
    \underline{Dr. Supeno Mardi Susiki Nugroho, S.T., M.T.} \\
    NIP. 197003131995121001
  
  \end{center}