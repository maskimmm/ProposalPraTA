\section{PENDAHULUAN}

\subsection{Latar Belakang}

% Ubah paragraf-paragraf berikut sesuai dengan latar belakang dari tugas akhir
Komunikasi, pada dasarnya merupakan aktivitas dasar manusia. Dengan adanya komunikasi, manusia dapat saling berhubungan dengan satu sama lain. Tulisan merupakan salah satu bentuk ragam komunikasi. Dengan adanya tulisan, suatu ide atau gagasan dapat dituangkan dan disampaikan kepada pembaca secara asinkronus serta dapat diabadikan. Secara umum, ragam tulisan dibagi menjadi 2 yaitu tulisan cetak pada dokumen dan tulisan tangan.\par
Segmentasi dokumen gambar menjadi suatu bentuk kalimat merupakan Langkah penting untuk memahami suatu dokumen. Tidak seperti dokumen cetak, segmentasi pada dokumen bertulisan tangan masih merupakan suatu hal yang menantang karena memiliki ukuran spacing yang tidak menentu antar hurufnya serta memiliki variasi bentuk gaya tulisan \citep*{ryu2015word}. Tulisan cetak pada dokumen merupakan tulisan yang pengaturan dan gaya penulisannya diatur dan dikenali oleh program computer.\par
\textit{Optical Character Recognition (OCR)} adalah proses konversi gambar huruf menjadi karakter ASCII yang dikenali oleh komputer. Walaupun diklaim memiliki akurasi hingga 99\%, \textit{Optical Character Recognition (OCR)} yang ada saat ini memiliki penurunan akurasi ketika dihadapkan kepada gambar dengan kualitas rendah seperti \textit{noise} gambar, kualitas cetakan rendah, karakter berdekatan \citep*{ImageMalu2001approachtch}, dan karakter dengan variasi yang tidak umum (tulisan tangan).\par
Papan tulis merupakan suatu media yang biasa digunakan untuk menuangkan tulisan, ide, ataupun gagasan dalam proses belajar dan mengajar. Seiring dengan perkembangan teknologi, berbagai teknologi \textit{Internet of Things} diterapkan pada papan tulis sehingga memiliki fitur tambahan yang dapat mempermudah proses belajar dan mengajar. Telah banyak teknologi yang dapat memproyeksikan gambar atau tulisan pada computer ke papan tulis pintar \citep*{kellerman2018smart}. Namun, belum ada teknologi yang mampu mengenali tulisan tangan pada papan tulis pintar.



\subsection{Permasalahan}

% Ubah paragraf berikut sesuai dengan permasalahan dari tugas akhir
Permasalahan yang didapat yaitu diperlukannya suatu metode untuk mendeteksi dan klasifikasi teks huruf balok untuk diterapkan pada alat \textit{Smart Whiteboard.} 

\subsection{Penelitian Terkait}
% Ubah paragraf berikut sesuai dengan penelitian lain yang terkait dengan tugas akhir

\subsubsection{Penelitian Berkaitan dengan \textit{Smart Whiteboard}}
Kellerman et al. \citep*{kellerman2018smart} mencoba untuk menyediakan suatu cara alternatif dan terjangkau pada papan tulis atau slides agar bisa mendapat interaksi lebih dari murid serta untuk meningkatkan efisiensi dari pengajaran. Pada penelitiannya, peneliti membuat sebuah papan tulis interaktif menggunakan Nintendo Wii \textit{Remote} dan \textit{PC Suite}. \textit{Software Suite} yang dikembangkan memungkinkan tampilan PC apapun dapat digunakan sebagai papan tulis interaktif. Sistem yang dibangun memiliki fungsi yang diperlukan untuk menciptakan sarana pembelajaran yang lebih baik dan lebih berteknologi, serta memberi pengguna dan siswa alat tambahan untuk menunjang Pendidikan interaktif. \textit{PC Suite} dibuat seramah mungkin sehingga dapat digunakan dengan mudah pada komputer standar.\

\subsubsection{Penelitian Berkaitan dengan \textit{Word Detection}}
Arun et al. \citep*{arun2019handwritten} menyajikan pendekatan sederhana untuk segmentasi huruf kata tulisan tangan menggunakan pendekatan bounding box dan pendekatan berbasis pixel. Segmentasi huruf tulisan tangan merupakan proses yang menantang karena gaya penulisan yang bervariasi. Kata-kata tulisan tangan yang tidak bersentuhan disegmentasikan dengan pendekatan bounding box dan kata-kata tulisan tangan yang bersentuhan disegmentasi menggunakan pendekatan pixel. Paper ini mencapai tingkat segmentasi hingga 94.45\% dan tingkat pengenalan 85.89\% dengan skema training dan testing 50-50\%.

\subsubsection{Penelitian Berkaitan dengan YOLO \textit{Object Detection}}
Karlina dan Indarti \citep*{karlina2020pengenalan} membuat pengenalan objek Makanan cepat saji dari \textit{video} dan \textit{real time webcam} menggunakan metode \textit{deep learning. You Only Look Once (YOLO)} merupakan model \textit{deep learning} yang digunakan untuk pengenalan objek. Jumlah data yang digunakan terdiri dari 468 gambar yang terdiri dari 3 jenis Makanan cepat saji. Nilai avg loss pada model akhir yang dibangun yaitu 4.6\%, nilai validasi mAP 100\%, serta akurasi akhir berkisar antara 63\% sampai 100\%.

\subsection{Gap Penelitian}
Pada penelitian berkaitan dengan \textit{smart whiteboard} \citep*{kellerman2018smart} telah dibuat \textit{smart whiteboard} yang dapat mendeteksi huruf yang dibuat dengan alat, namun tidak dapat mendeteksi tulisan yang dibuat pada papan tulis. Kemudian, pada penelitian berkaitan dengan \textit{word detection} \citep*{ryu2015word} \citep*{arun2019handwritten} telah dibuat algoritma pendeteksi huruf menggunakan pendekatan textit{structure learning, bounding box,} dan \textit{pixel based,} namun tidak menggunakan metode \textit{You Only Look Once (YOLO).} Sedangkan pada penelitian berkaitan dengan \textit{YOLO Object Detection} \citep*{karlina2020pengenalan} YOLO digunakan untuk deteksi objek berupa Makanan cepat saji, namun tidak digunakan untuk deteksi huruf balok pada papan tulis pintar.

\subsection{Tujuan Penelitian}
Tujuan dari dibuatnya tugas akhir ini yaitu untuk membuat program komputer yang dapat melakukan pengenalan teks huruf balok sehingga dapat diimplementasikan pada alat \textit{Smart Whiteboard.}