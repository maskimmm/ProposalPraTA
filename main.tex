% Pengaturan ukuran teks dan jenis dokumen
\documentclass[11pt]{article}

% Pengaturan ukuran halaman dan margin
\usepackage[a4paper,top=30mm,left=30mm,right=20mm,bottom=20mm]{geometry}

% Pengaturan ukuran spasi
\usepackage[singlespacing]{setspace}

% Judul dokumen
\title{Proposal Pra Tugas Akhir ITS}
\author{Rifqi Alhakim Hariyantoputera}

% Pengaturan format bahasa
\usepackage[indonesian]{babel}

% Pengaturan detail pada file PDF
\usepackage[pdfauthor={\@author},bookmarksnumbered,pdfborder={0 0 0}]{hyperref}

% Pengaturan jenis karakter
\usepackage[utf8]{inputenc}

% Pengaturan ukuran indentasi
\setlength{\parindent}{2em}

% Package lainnya
\usepackage{etoolbox} % Mengubah fungsi default
\usepackage{enumitem} % Pembuatan list
\usepackage{lipsum} % Pembuatan template kalimat
\usepackage{graphicx} % Input gambar
\usepackage{longtable} % Pembuatan tabel
\usepackage[table,xcdraw]{xcolor} % Pewarnaan tabel
\usepackage[numbers]{natbib} % Kutipan artikel
\usepackage{changepage} % Pembuatan teks kolom
\usepackage{multicol} % Pembuatan kolom ganda
\usepackage{multirow} % Pembuatan baris ganda
\usepackage{float}
\usepackage{outlines}

% Pengaturan format judul bab
\usepackage{titlesec}
\renewcommand{\thesection}{\arabic{section}}
\titleformat*{\section}{\normalsize\bfseries}
\titlespacing{\section}{0ex}{3ex}{1.5ex}
\titleformat*{\subsection}{\normalsize\bfseries}
\titlespacing{\subsection}{0ex}{3ex}{1.5ex}
\titleformat*{\subsubsection}{\normalsize\bfseries}
\titlespacing{\subsubsection}{5ex}{1ex}{1ex}
% \titleformat*{\subsection}{\normalsize\bfseries}
% \titlespacing{\subsection}{0ex}{3ex}{1.5ex}

% Isi keseluruhan dokumen
\begin{document}

  % Menonaktifkan penomoran halaman
  \pagenumbering{gobble}

  % Lembar pengesahan
  \begin{flushleft}
    % Ubah kalimat berikut sesuai dengan nama departemen dan fakultas
    \textbf{Departemen Teknik Komputer - FTEIC}\\
    \textbf{Institut Teknologi Sepuluh Nopember}\\
  \end{flushleft}
  
  \begin{center}
    % Ubah detail mata kuliah berikut sesuai dengan yang ditentukan oleh departemen
    \underline{\textbf{EC184701 - PRA TUGAS AKHIR (2 SKS)}}
  \end{center}
  
  \begin{adjustwidth}{-0.2cm}{}
    \begin{tabular}{lcp{0.7\linewidth}}
  
      % Ubah kalimat-kalimat berikut sesuai dengan nama dan NRP mahasiswa
      Nama Mahasiswa &:& Fathullah Auzan Setyo Laksono \\
      Nomor Pokok &:& 07211840000053 \\
  
      % Ubah kalimat berikut sesuai dengan semester pengajuan proposal
      Semester &:& Ganjil 2021/2022 \\
  
      % Ubah kalimat-kalimat berikut sesuai dengan nama-nama dosen pembimbing
      Dosen Pembimbing &:& 1. Reza Fuad Rachmadi, S.T., M.T., Ph.D. \\
      & & 2. Dr. Eko Mulyanto Yuniarno, S.T., M.T. \\
  
      % Ubah kalimat berikut sesuai dengan judul tugas akhir
      Judul Tugas Akhir &:& \textbf{Estimasi Umur, Gender dan Etnik Menggunakan} \\
      & & \textbf{Covolutional Neural Network Berbasis Citra Wajah} \\
  
      Uraian Tugas Akhir &:& \\
    \end{tabular}
  \end{adjustwidth}
  
  % Ubah paragraf berikut sesuai dengan uraian dari tugas akhir
  Fitur wajah seperti identifikasi umur, gender dan etnik dapat sangat berguna dalam banyak pengimplementasian ilmu seperti pengamatan visual, diagnosa medis, sistem interaksi komputer manusia, biometric, pengumpulan informasi, penegakan hukum, pemasaran dan banyak lainnya. Dimana sebagian besar data mengenai fitur wajah tersebut masih diambil secara manual melalui survei ataupun pengamatan pada banyak individu. Berdasarkan World Population Clock pada websitenya, di dunia terdapat lebih dari 7 miliar orang yang tersebar di berbagai macam pulau dan benua. Jumlah tersebut masih terus bertambah sampai sekarang. Dimana di setiap benua dan negara tersebut terdapat berbagai karakteristik dan ciri manusia yang berbeda dengan kata lain Etnik yang berbeda-beda. Dengan banyaknya jumlah penduduk dan keberagamannya tersebut, jika data fitur wajah diambil secara manual akan memakan waktu dan tenaga yang banyak. Oleh karena itu perlu dibuat suatu sistem yang dapat mengestimasi umur, gender dan etnik serta mengyimpan penghitungan datanya untuk mempermudah pengumpulan data. Dimana kamera akan menangkap gambar dari seseorang dan dilakukan proses estimasi umur, gender dan etnik yang kemudian datanya disimpan untuk digunakan kedepannya.
  \vspace{1ex}
  
  \begin{flushright}
    % Ubah kalimat berikut sesuai dengan tempat, bulan, dan tahun penulisan
    Surabaya, Desember 2021
  \end{flushright}
  \vspace{1ex}
  
  \begin{center}
  
    \begin{multicols}{2}
  
      Dosen Pembimbing 1
      \vspace{12ex}
  
      % Ubah kalimat-kalimat berikut sesuai dengan nama dan NIP dosen pembimbing pertama
      \underline{Dr. I Ketut Eddy Purnama, S.T., M.T.} \\
      NIP. 196907301995121001
  
      \columnbreak
  
      Dosen Pembimbing 2
      \vspace{12ex}
  
      % Ubah kalimat-kalimat berikut sesuai dengan nama dan NIP dosen pembimbing kedua
      \underline{Reza Fuad Rachmadi S.T., M.T., Ph.D.} \\
      NIP. 198504032012121001
  
    \end{multicols}
    \vspace{6ex}
  
    Mengetahui, \\
    % Ubah kalimat berikut sesuai dengan jabatan kepala departemen
    Kepala Departemen Teknik Komputer FTEIC - ITS
    \vspace{12ex}
  
    % Ubah kalimat-kalimat berikut sesuai dengan nama dan NIP kepala departemen
    \underline{Dr. Supeno Mardi Susiki Nugroho, S.T., M.T.} \\
    NIP. 197003131995121001
  
  \end{center}
  \newpage

  \begin{center}
    % Ubah judul
    \textbf{\textit{Smart Whiteboard:} Pengenalan Teks Huruf Balok Menggunakan \textit{You Only Look Once (YOLO)}}
  \end{center}

  % Konten pendahuluan
  \section{PENDAHULUAN}

\subsection{Latar Belakang}

% Ubah paragraf-paragraf berikut sesuai dengan latar belakang dari tugas akhir
Umur, gender dan etnik merupakan beberapa hal penting dalam wajah yang menentukan bagaimana seorang individu berinteraksi sosial. Setiap bahasa di dunia memiliki panggilan kehormatan yang berbeda-beda untuk pria dan wanita, perbedaan umur juga dapat menentukan bagaimana seseorang harus bersikap dengan orang yang lebih muda ataupun dengan yang lebih tua, sedangkan etnik juga dapat menentukan cara berbahasa dan berperilaku pada seseorang. Kebiasaan dan sikap tersebut sebagian besar tergantung pada kemampuan seseorang dalam memperkirakan atau mengestimasi individu tersebut melalui penampakan gender, umur dan etnik. Dimana identitas, ekspresi, gender, umur dan etnik disebut dengan fitur dalam wajah.
Selain itu, fitur dalam wajah juga sering digunakan di berbagai bidang, seperti di kepolisian untuk mencari pelaku tindak kriminal yang mengidentifkasi pelaku dari wajah. diagnosa medis yang menggunakan wajah untuk menentukan penanganan yang cocok untuk pasien. Di dunia bisnis fitur wajah juga digunakan dalam pembagian target pasar untu lebih meningkatkan proses bisnis.
Hal ini membuat identifikasi umur, gender dan etnik dapat sangat berguna dalam banyak pengimplementasian ilmu seperti pengamatan visual, diagnosa medis, sistem interaksi komputer manusia, biometric, pengumpulan informasi, penegakan hukum, pemasaran dan banyak lainnya. Dimana sebagian besar data mengenai fitur wajah tersebut masih diambil secara manual melalui survei ataupun pengamatan pada banyak individu.
Berdasarkan World Population Clock pada websitenya, di dunia terdapat lebih dari 7 miliar orang yang tersebar di berbagai macam pulau dan benua. Jumlah tersebut masih terus bertambah sampai sekarang. Dimana di setiap benua dan negara tersebut terdapat berbagai karakteristik dan ciri manusia yang berbeda dengan kata lain Etnik yang berbeda-beda. Dengan banyaknya jumlah penduduk dan keberagamannya tersebut, jika data fitur wajah diambil secara manual akan memakan waktu dan tenaga yang banyak.


\subsection{Permasalahan}

% Ubah paragraf berikut sesuai dengan permasalahan dari tugas akhir
Pengambilan data terkait fitur wajah terutama umur, gender dan etnis masih dilakukan secara manual yang membutuhkan waktu dan tenaga relatif banyak. Oleh karena itu, diperlukan model yang dapat mengestimasi umur, gender dan etnik dari individu untuk mempermudah proses pengambilan data.

\subsection{Penelitian Terkait}

% Ubah paragraf berikut sesuai dengan penelitian lain yang terkait dengan tugas akhir
Berberapa penelitian yang telah dilakukan terkait dengan judul Tugas Akhir ini antara lain dilakukan oleh A. Garain et al. GRANet A Deep Learning Model for Classification of Age and GenderFrom Facial Images. Dimana pada penelitian tersebut mencoba menggunakan beberapa dataset seperti wikipedia age dataset, FG-Net, AFAD, AduenceDB dataset dan UTKFace dataset serta menggunakan model yang arsitekturnya seperti Residual Attention Network dengan tambahan parameter “Gate” seperti pada Gated Residual Units (GRU’s). Yang berhasil melakukan prediksi umur dan gender dengan baik. Namun belum menggunakan pendeteksian etnik. Penelitian lainnya dilakukan oleh G. Guo et al. dengan judul Human Age Estimation What is the Influence Across Race and Gender yang menggunakan database MORPH-II dengan data gambar wajah sebanyak 55.000. Mereka membandingkan hasil estimasi umur antara individu dengan sesama etnik dan dengan yang berbeda etnik. Didapatkan tingkat eror yang signifikan pada percobaan estimasi individu  yang  berbeda etnik. Kemudian penelitian oleh M. Shin et al. Face Image-Based Age and Gender Estimation with Consideration of Ethnic Difference, pada penelitian ini menggunakan CNN dan SVM yang berfungsi memisahkan dua etnik menjadi Asia dan Non Asia yang kemudian hasilnya digunakan untuk mendeteksi umur dan gender dari wajah yang diberikan. Dihasilkan bahwa pemisahan etnik dapat meningkatkan keakuratan estimasi umur, namun tidak berpengaruh pada gender.

\subsection{Gap Penelitian}
Pada penelitian sebelumnya beberapa sudah menggunakan faktor etnis sebagai penentu estimasi umur dan gender, namun belum mengestimasikan ras dan juga masih merupakan program yang mendeteksi foto dari input yang berasal dari dataset bukan dari kamera secara langsung.

\subsection{Tujuan Penelitian}

% Ubah paragraf berikut sesuai dengan tujuan penelitian dari tugas akhir
Tujuan yang ingin dicapai dari Tugas Akhir ini adalah untuk mengembangkan sebuah model menggunakan Convolutional Neural Network yang dapat mengestimasi umur, gender dan etnik yang mempermudah proses pengumpulan dan pengambilan data terkait umur, gender dan etnik.

  % Konten tinjauan pustaka
  \section{TINJAUAN PUSTAKA}

\subsection{\textit{Convolutional  Neural Network (CNN)}}
CNN merupakan algoritma \textit{deep learning} yang mampu mengambil masukan berupa gambar, menetapkan prioritas untuk berbagai aspek/objek dalam gambar dan mampu membedakan satu sama lain. Tahapan \textit{pre-processing} yang dibutuhkan CNN lebih sedikit jika dibandingkan dengan algoritma klasifikasi lainnya \citep*{towardsDS}.

% Contoh penggunaan referensi dari pustaka
% Newton pernah merumuskan \citep{Newton1687} bahwa \lipsum[8]
% Contoh penggunaan referensi dari persamaan
% Kemudian menjadi persamaan seperti pada persamaan \ref{eq:FirstLaw}.

\subsection{\textit{You Only Look Once (YOLO)}}
YOLO merupakan salah satu arsitektur dari CNN yang dioptimasi untuk mendeteksi objek pada gambar. Arsitektur YOLO sangat cepat apabila dibandingkan dengan arsitektur pengenalan objek lainnya \citep*{jeong2018image}. 

% % input gambar
% \begin{figure} [H] \centering
%     % Nama dari file gambar yang diinputkan
%     \includegraphics[scale=0.6]{gambar/umur.png}
%     % Keterangan gambar yang diinputkan
%     \caption{Kategori umur menurut Depkes. RI (2009)}
%     % Label referensi dari gambar yang diinputkan
%     \label{fig:Umur}
% \end{figure}

\subsection{\textit{Word Segmentation}}
Pengenalan tulisan tangan merupakan Teknik untuk menginterpretasikan tulisan tangan kedalam bentuk digital. Proses pengenalan tulisan tangan dapat diperoleh dengan 2 cara yaitu dengan mengonversi otomatis karakter pada saat ditulis pada layar sentuh dengan pena digital dan cara lain yaitu dengan melakukan pengambilan gambar serta pemrosesan gambar pada suatu teks yang ingin dikenali [8]. Pada proses segmentasi huruf, mulanya dokumen gambar disegmentasi kedalam baris-baris teks. Kemudian, algoritma segmentasi huruf diterapkan pada satu baris teks tersebut. Pada satu baris teks tersebut, secara umum proses segmentasi huruf konvensional menjalankan algoritma yang terdiri dari 2 tahapan yaitu: ekstraksi kandidat huruf berdasarkan pemisah huruf dan dilanjut dengan klasifikasi kandidat huruf \citep*{ryu2015word}.

% \subsection{Deep Learning}
%  Deep Learning merupakan artificial neural network yang memiliki banyak layer dan synapse weight. 
%  Deep learning dapat menemukan relasi tersembunyi atau pola yang rumit antara input dan output, yang 
%  tidak dapat diselesaikan menggunakan multilayer perceptron. Keuntungan  utama  deep  learning  yaitu 
%  mampu merubah data dari nolinearly separable menjadi linearly separable melalui serangkaian transformasi 
%  (hidden layers). Selain itu, deep learning juga mampu mencari decision boundary yang berbentuk non-linier
%  , serta mengsimulasikan interaksi non-linier antar fitur. Jadi, input ditransformasikan secara 
%  non-linier sampai akhirnya pada output, berbentuk distribusi class-assignment\citep{DeepLearning}.

%  \begin{figure} [H] \centering
%     % Nama dari file gambar yang diinputkan
%     \includegraphics[scale=0.6]{gambar/deeplearning.png}
%     % Keterangan gambar yang diinputkan
%     \caption{Deep Learning 4 layer}
%     % Label referensi dari gambar yang diinputkan
%     \label{fig:Deep Learning}
% \end{figure}

% \subsection{Convolutional Neural Network (CNN)}
% Convolutional Neural Network (CNN) merupakan cabang dari Multilayer Perceptron (MLP) yang digunakan untuk
% mengolah data dua dimensi. CNN memiliki kedalaman jaringan yang tinggi sehingga CNN termasuk dalam jenis
% Deep Neural Network. Perbedaan CNN dengan MLP terdapat pada neuron dimana pada MLP setiap neuron hanya
% berukuran satu dimensi, sedangkan CNN setiap neuronnya berukuran dua dimensi. Pada CNN, operasi linier
% menggunakan operasi konvolusi\citep{CNN}.

% \subsection{Image Processing}
% Image Processing atau Pengolahan Citra merupakan teknik dalam pemrosesan gambar dengan input berupa 
% citra dua dimensi yang bertujuan untuk menyempurnakan citra atau mendapatkan informasi yang berguna 
% untuk diolah menjadi beberapa keputusan. Dalam operasi pemrosesan citra, operasi yang sering dilakukan 
% dalam format gambar grayscale. Gambar grayscale didapatkan dari pemrosesan gambar berwarna yang 
% didekomposisi menjadi komponen merah (R), hijau (G) dan biru (B) yang diproses secara independen sebagai 
% gambar grayscale. Image Processing terbagi menjadi dalam tiga tingkatan\citep{ImageProcesing}:
%     \begin{enumerate}
%         \item Low-Level Image Processing \\
%         Low-Level Image Processing merupakan operasi sederhana dalam pengolahan gambar dimana input dan 
%         output berupa gambar. Contoh: contrast enchancement dan noise reduction.
%         \item Mid-Level Image Processing \\
%         Mid-Level Image Processing merupakan operasi pengolahan gambar yang melibatkan ekstrasi atribut dari 
%         gambar input. Contoh: edges, contours dan regions.
%         \item High-Level Image Processing \\
%         High-Level Image Processing merupakan merupakan kategoriyang melibatkan pemrosesan gambar kompleks 
%         yang terkait dengan analisis dan interpretasi konten dalam sebuah keadaan untuk pengambilan keputusan.
%     \end{enumerate}




  % Konten metodologi
  \section{METODOLOGI}

% Ubah konten-konten berikut sesuai dengan isi dari metodologi

\subsection{Data dan Peralatan}

Berikut merupakan data dan perlatan yang mendukung pengerjaan Tugas Akhir ini.
\begin{itemize}
   \item [a.] UTKFace Dataset \\
   UTKFace dataset  merupakan dataset wajah dengan rentan umur yang panjang (dari 0 hingga 116 tahun). 
   Terdiri atas lebih dari 20.000 gambar wajah dengan anotasi umur, jenis kelamin dan etnik. Gambar wajah
   dalam dataset memiliki berbagai pose, ekspresi wajah, resolusi dan banyak lagi. Dataset ini dapat 
   digunakan untuk berbagai jenis tugas seperti deteksi wajah, perkiraan umur, pengurangan umur dan 
   lainnya[14]. Untuk waktu pengaksesan dataset ini adalah 10 Oktober 2021 pada link 
   www.kaggle.com/jangedoo/utkface-new.
    \begin{figure} [H] \centering
      % Nama dari file gambar yang diinputkan
      \includegraphics[scale=0.2]{gambar/UTKFace.png}
      % Keterangan gambar yang diinputkan
      \caption{Dataset UTKFace}
      % Label referensi dari gambar yang diinputkan
      \label{fig:UTKFace}
    \end{figure}

   \item [b.] Laptop \\
   Personal computer portable dengan spesifikasi prosesor intel i5 generasi lima, GPU nvidia Geforce 930M, 
   RAM 8GB dan storage 750GB akan digunakan untuk pembuatan model dan pencobaan model yang sudah jadi 
   menggunakan kamera yang tersedia pada perangkat.

   \item [c.] Google Collab \\
   Merupakan sebuah website yang dimiliki oleh Google yang dapat digunakan untuk membuat program dan 
   menjalankannya. Menyediakan beberapa pilihan untuk menjalankan program melalui koneksi internet dan 
   tidak membebani kerja komputer. Dalam tugas akhir ini berfungsi sebagai tempat menjalankan program 
   training dataset dan juga pembuatan model menggunakan GPU dalam cloud yang disediakan Google Collab.
\end{itemize}
   

\subsection{Metodologi Penelitian}
Metodologi yang digunakan dalam pengerjaan Tugas Akhir ini adalah sebagai berikut.
    % Contoh input gambar dengan format *.jpg
    \begin{figure} [H] \centering
      % Nama dari file gambar yang diinputkan
      \includegraphics[scale=0.5]{gambar/Metodologi.png}
      % Keterangan gambar yang diinputkan
      \caption{Diagram blok metodologi}
      % Label referensi dari gambar yang diinputkan
      \label{fig:Metodologi}
    \end{figure}

\begin{enumerate}
   \item \textbf{Pemrosesan Dataset} \\
   Pada tahap ini dataset yang digunakan aakn dicek dan dibagi berdasarkan training, validation dan 
   testing, serta menyiapkan dataset untuk digunakan pada pembuatan model nantinya.
   \item \textbf{Pemilihan Model} \\
   Memilih model atau arsitektur Convolutional Neural Network atau CNN yang tepat sesuai dengan UTKFace 
   dataset yang telah disediakan sebelumnya dan membangun model untuk keperluan training.
   \item \textbf{Training} \\
   Pada tahap ini dataset akan digunakan untuk melatih komputer dengan cara mengolah gambar dan anotasi 
   yang telah dibuat sehingga terbentuk pola atau karakteristik dari masing masing kelas yang akan 
   menjadi bahan pertimbangan komputer dalam mencapai sebuah keputusan atau melakukan prediksi. 
   Proses training akan dilakukan menggunakan Covolutional Neural Network atau CNN pada citra wajah 
   yang diberikan.
   \item \textbf{Tuning} \\
   Model yang sudah jadi akan dievaluasi dan dilakukan pengaturan lagi untuk meningkatkan peforma dan 
   keakuratan model.
   \item \textbf{Pengujian dan Analisa} \\
   Sistem yang sudah jadi akan diuji pada dataset yang telah disiapkan untuk testing dan juga akan dicoba 
   pada kamera di wilayah ramai orang untuk mengumpulkan data. Kemudian data yang diperoleh dianalisa 
   dan dicatat untuk keperluan pembuatan laporan nantinya.
\end{enumerate}

  % Konten lainnya
  \section{HASIL YANG DIHARAPKAN}

\subsection{Hasil yang Diharapkan dari Penelitian}

Penelitian ini diharapkan dapat menghasilkan sebuah sistem yang dapat mengklasifikasikan 4 tipe kondisi otak, yaitu: Normal, Alzheimer, Tumor dan Stroke dengan menggunakan metode 3D CNN yang dapat digunakan untuk membantu dokter radiologi dalam membuat diagnosa.

\subsection{Hasil Pendahuluan}

Sampai saat ini, kami telah \lipsum[16]

\section{RENCANA KERJA}

% Ubah tabel berikut sesuai dengan isi dari rencana kerja
\newcommand{\w}{}
\newcommand{\G}{\cellcolor{gray}}
\begin{table}[h!]
  \begin{tabular}{|p{3.5cm}|c|c|c|c|c|c|c|c|c|c|c|c|c|c|c|c|}

    \hline
    \multirow{2}{*}{Kegiatan} & \multicolumn{16}{|c|}{Minggu} \\
    \cline{2-17} &
    1 & 2 & 3 & 4 & 5 & 6 & 7 & 8 & 9 & 10 & 11 & 12 & 13 & 14 & 15 & 16 \\
    \hline

    % Gunakan \G untuk mengisi sel dan \w untuk mengosongkan sel
    Data Collection &
    \G & \G & \w & \w & \w & \w & \w & \w & \w & \w & \w & \w & \w & \w & \w & \w \\
    \hline

    Data Pre-Processing &
    \w & \G & \G & \G & \w & \w & \w & \w & \w & \w & \w & \w & \w & \w & \w & \w \\
    \hline

    Pembuatan Model &
    \w & \w & \w & \w & \G & \G & \G & \G & \G & \G & \G & \G & \G & \G & \w & \w \\
    \hline

    Training Model &
    \w & \w & \w & \w & \w & \w & \G & \G & \G & \G & \G & \G & \G & \G & \w & \w \\
    \hline
    
    Evaluasi Model &
    \w & \w & \w & \w & \w & \w & \w & \G & \G & \G & \G & \G & \G & \G & \w & \w \\
    \hline
    
    Pembuatan Laporan &
    \w & \w & \w & \w & \w & \w & \w & \w & \w & \w & \w & \w & \w & \w & \G & \G \\
    \hline

  \end{tabular}
\end{table}

  % Daftar pustaka
  \renewcommand\bibname{DAFTAR PUSTAKA}
  \addcontentsline{toc}{chapter}{\bibname}
  \bibliographystyle{unsrtnat}
  \bibliography{dafpus/pustaka.bib}
  \cleardoublepage
  %\section{DAFTAR PUSTAKA}
  %\renewcommand\refname{}
  %\vspace{-2ex}
  %\bibliographystyle{unsrtnat}
  %\bibliography{pustaka/pustaka.bib}

\end{document}